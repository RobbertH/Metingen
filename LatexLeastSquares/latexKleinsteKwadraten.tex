\documentclass{article}
\usepackage[utf8]{inputenc}
\usepackage[portrait, margin=1.5in]{geometry}
\usepackage{amsmath}

\title{Least Squares approximation for humidity sensor}
\author{Robbert Hofman}
\date{July 2016}

\begin{document}

\maketitle

\section{Introduction}
In the table below, $hum$ stands for the humidity measured by a gehaka measuring device that is calibrated correctly. 
This is the reference humidity that we are trying to calibrate at. 
The variable $cycles$ is the amount of cycles measured by our device.
These are purely fictive numbers, solely for the purpose of providing an example.

\begin{center}
\begin{tabular}{ c c c }
\hline
 Sample number & Cycles & Humidity \\ 
\hline
 0 & 3497 & 11.1  \\  
 1 & 3994 & 11.9 \\
 2 & 4511 & 13.0 \\
 3 & 4913 & 14.1 \\ 
 ... & ... & ... \\ 
 n & 2900 & 10.1
\end{tabular}
\end{center}

We will attempt to make a function of the following form, making the $error$ term as small as possible
\begin{equation}
\label{formeq}
f(cycles) = hum + error
\end{equation}

\section{Linear regression}
For linear regression, eq. \ref{formeq} looks like this:
\begin{equation}
\label{formlin}
a*cycles + b = hum + error
\end{equation}

The function that yields the distance between the real and estimated humidities is denoted as follows
\begin{equation}
	\begin{split}
		f(a,b) = [hum_{0} - (a*cycles_{0} + b)]^{2} \\
		+ [hum_{1} - (a*cycles_{1} + b)]^{2} \\
	 + ... +  [hum_{n} - (a*cycles_{n} + b)]^{2}
	\end{split}
\end{equation}

\begin{equation}
	\begin{split}
	f(a,b) = hum_{0}^{2} - 2*hum_{0}*(a*cycles_{0}+b) + a^{2}*cycles_{0}^{2} + 2*a*b*cycles_{0}+b^{2} \\
+ hum_{1}^{2} - 2*hum_{1}*(a*cycles_{1}+b) + a^{2}*cycles_{1}^{2} + 2*a*b*cycles_{1}+b^{2} \\
+ ...  + hum_{n}^{2} - 2*hum_{0}*(a*cycles_{n}+b) + a^{2}*cycles_{n}^{2} + 2*a*b*cycles_{n}+b^{2}
	\end{split}
\end{equation}


To minimize this distance function, we need to set all first partial derivatives to zero: 
\begin{subequations}
	\begin{equation}
		\frac{\partial}{\partial a}f(a,b) = 0 
	\end{equation}
	\begin{equation}
		\frac{\partial}{\partial b}f(a,b) = 0 
	\end{equation}
\end{subequations}

\begin{subequations}
	\begin{equation}
		\begin{split}
		\frac{\partial}{\partial a}f(a,b) = -2*hum_{0}*cycles_{0} + 2*a*cycles_{0}^{2} + 2*b*cycles_{0} \\
		 -2*hum_{1}*cycles_{1} + 2*a*cycles_{1}^{2} + 2*b*cycles_{1} \\
		+ \dots
 		-2*hum_{n}*cycles_{n} + 2*a*cycles_{n}^{2} + 2*b*cycles_{n} \\
		= 0
		\end{split}
	\end{equation}
	\begin{equation}
		\begin{split}
			\frac{\partial}{\partial b}f(a,b) = -2*hum_{0} + 2*a*cylos_{0} + 2*b \\
			-2*hum_{1} + 2*a*cylos_{1} + 2*b \\
			+ \dots
			-2*hum_{n} + 2*a*cylos_{n} + 2*b \\
			= 0
		\end{split}
	\end{equation}
\end{subequations}

Or equivalently,
\begin{subequations}
	\begin{equation}
	\label{dfda}
		\frac{\partial}{\partial a}f(a,b) = -2*\sum_{i=0}^{n}{hum_{i}*cycles_{i}} + 2*a*\sum_{i=0}^{n}{cycles_{i}^{2}} + 2*b*\sum_{i=0}^{n}{cycles_{i}} = 0
	\end{equation}
	\begin{equation}
	\label{dfdb}
		\frac{\partial}{\partial a}f(a,b) = -2*\sum_{i=0}^{n}{hum_{i}} + 2*a*\sum_{i=0}^{n}{cycles_{i}} + 2*n*b = 0
	\end{equation}
\end{subequations}

Or equivalently, from \ref{dfdb},
\begin{subequations}
	\begin{equation}
	\label{b}
		b = \frac{\sum_{i=0}^{n}{hum_{i}}}{n} - a * \frac{\sum_{i=0}^{n}{cycles_{i}}}{n} 
	\end{equation}
\end{subequations}

Note that this is also the average $hum$ - a times the average $cycles$, which is what we'd expect to see, as we are trying to get to a function of the same form as eq. \ref{formlin}.

Then, from \ref{dfda}, we can find a:
\begin{equation}
	-\sum_{i=0}^{n}{hum_{i}*cycles_{i}} + a*\sum_{i=0}^{n}{cycles_{i}^{2}} + b*\sum_{i=0}^{n}{cycles_{i}} = 0
\end{equation}
Substituting b for eq. \ref{b}, we find
\begin{equation}
	\begin{split}
		-\sum_{i=0}^{n}{hum_{i}*cycles_{i}} + a*\sum_{i=0}^{n}{cycles_{i}^{2}} \\ 
		+ \left( \frac{\sum_{i=0}^{n}{hum_{i}}}{n} - a * \frac{\sum_{i=0}^{n}{cycles_{i}}}{n}\right)*\sum_{i=0}^{n}{cycles_{i}} = 0
	\end{split}
\end{equation}
Or
\begin{equation}
	\begin{split}
		-\sum_{i=0}^{n}{hum_{i}*cycles_{i}}  + \frac{\sum_{i=0}^{n}{hum_{i}}}{n}*\sum_{i=0}^{n}{cycles_{i}} \\ 
		= - a*\sum_{i=0}^{n}{cycles_{i}^{2}} + a*\frac{(\sum_{i=0}^{n}{cycles_{i}})^{2} }{n}
	\end{split}
\end{equation}
Which finally yields the equation for a:
\begin{equation}
	a = \frac{	-\sum_{i=0}^{n}{hum_{i}*cycles_{i}}  + \frac{\sum_{i=0}^{n}{hum_{i}}}{n}*\sum_{i=0}^{n}{cycles_{i}}  }{-\sum_{i=0}^{n}{cycles_{i}^{2}}+\frac{(\sum_{i=0}^{n}{cycles_{i}})^{2} }{n} }
\end{equation}

\section{Quadratic regression}

\begin{equation}
	a*cycles^{2}+b*cycl+c=hum+err
\end{equation}


\section{Exponential regression}
\begin{equation}
	a*e^{b*cycles}+c
\end{equation}
\begin{equation}
	f(a,b,c) = \sum_{i=0}^{n}{\left(hum_{i}-(a*e^{b*cycles_{i}}+c)\right)^2}
\end{equation}
\begin{equation}
	f(a,b,c) = \sum_{i=0}^{n}{\left(hum_{i}^{2}-2*hum_{i}*(a*e^{b*cycles_{i}}+c)+a^{2}*e^{2*b*cycles}+2*a*c*e^{b*cycles} + c^{2}\right)}
\end{equation}
\begin{subequations}
	\begin{equation}
		\frac{\partial{}}{\partial a}f = 0
	\end{equation}
	\begin{equation}
		\frac{\partial{}}{\partial b}f = 0
	\end{equation}
	\begin{equation}
		\frac{\partial{}}{\partial c}f = 0
	\end{equation}
\end{subequations}
\begin{subequations}
	\begin{equation}
		\frac{\partial{}}{\partial a}f = -2*hum*e^{b*cycl}+2*a*e^{2*b*cycles}+2*c*e^{b*cycles} = 0
	\end{equation}
	\begin{equation}
		\frac{\partial{}}{\partial b}f = -2*a*hum*cycl*e^{b*cycl}+2*a^{2}*cycles*e^{2*b*cycles}+2*a*cycles*c*e^{b*cycles}=0
	\end{equation}
	\begin{equation}
		\frac{\partial{}}{\partial c}f = -2*hum+2*a*e^{b*cyxles}+2*c=0
	\end{equation}
\end{subequations}


\section{Conclusion}
For linear approximation of the form 
$a*cycles + b = hum + error$
The coefficients a and b can be determined as follows:
$$	a = \frac{	-\sum_{i=0}^{n}{hum_{i}*cycles_{i}}  + \frac{\sum_{i=0}^{n}{hum_{i}}}{n}*\sum_{i=0}^{n}{cycles_{i}}  }{-\sum_{i=0}^{n}{cycles_{i}^{2}}+\frac{(\sum_{i=0}^{n}{cycles_{i}})^{2} }{n} } $$
$$ b = \frac{\sum_{i=0}^{n}{hum_{i}}}{n} - a * \frac{\sum_{i=0}^{n}{cycles_{i}}}{n} $$
%\bibliographystyle{plain}
%\bibliography{references}
\end{document}
