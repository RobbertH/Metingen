\documentclass{article}
\usepackage[utf8]{inputenc}
\usepackage[portrait, margin=1.5in]{geometry}
\usepackage{amsmath}

\title{Least Squares approximation for humidity sensor}
\author{Robbert Hofman}
\date{July 2016}

\begin{document}

\maketitle

\section{Introduction}
In the table below, $hum$ stands for the humidity measured by a gehaka measuring device that is calibrated correctly. 
This is the reference humidity that we are trying to calibrate on. 
The variable $cycles$ is the amount of cycles measured by our device.

\begin{center}
\begin{tabular}{ c c c }
\hline
 Sample number & Cycles & Humidity \\ 
\hline
 0 & 3497 & 11.1  \\  
 1 & 3994 & 11.9 \\
 2 & 4511 & 13.0 \\
 3 & 4913 & 14.1 \\ 
 ... & ... & ... \\ 
 n & 2900 & 10.1
\end{tabular}
\end{center}

\begin{equation}
y_n(x) = \sum_{k=0}^n {'} a_k T_k(x)
\end{equation}
We will attempt to make a function of the following form, making the $error$ term as small as possible
\begin{equation}
a*cycles + b = hum + error
\end{equation}
The function that yealds the distance between the real and estimated humidities is denoted as follows
\begin{equation}
f(a,b) = [hum_{0} - (a*cyclos_{0} + b)]^{2} + [hum_{1} - (a*cyclos_{1} + b)]^{2}
	 + ... +  [hum_{n} - (a*cyclos_{n} + b)]^{2}
\end{equation}

\begin{equation}
	f(a,b) = hum_{0}^{2} - 2*hum_{0}*(a*cyclos_{0}+b) + a^{2}*cyclos_{0}^{2} + 2*a*b*cyclos_{0}+b^{2} \newline
+ hum_{1}^{2} - 2*hum_{1}*(a*cyclos_{1}+b) + a^{2}*cyclos_{1}^{2} + 2*a*b*cyclos_{1}+b^{2}
+ ... 
+ hum_{n}^{2} - 2*hum_{0}*(a*cyclos_{n}+b) + a^{2}*cyclos_{n}^{2} + 2*a*b*cyclos_{n}+b^{2}
\end{equation}

To minimize this distance function, we need to set all first partial derivatives to zero: 
\begin{subequations}
	\begin{equation}
		\frac{\partial}{\partial a}f(a,b) = 0 
	\end{equation}
	\begin{equation}
		\frac{\partial}{\partial b}f(a,b) = 0 
	\end{equation}
\end{subequations}

\begin{subequations}
	\begin{equation}
		\begin{split}
		\frac{\partial}{\partial a}f(a,b) = -2*hum_{0}*cyclos_{0} + 2*a*cyclos_{0}^{2} + 2*b*cyclos_{0} \\
		 -2*hum_{1}*cyclos_{1} + 2*a*cyclos_{1}^{2} + 2*b*cyclos_{1} \\
		+ \dots
 		-2*hum_{n}*cyclos_{n} + 2*a*cyclos_{n}^{2} + 2*b*cyclos_{n} \\
		= 0
		\end{split}
	\end{equation}
	\begin{equation}
		\begin{split}
			\frac{\partial}{\partial b}f(a,b) = -2*hum_{0} + 2*a*cylos_{0} + 2*b \\
			-2*hum_{1} + 2*a*cylos_{1} + 2*b \\
			+ \dots
			-2*hum_{n} + 2*a*cylos_{n} + 2*b \\
			= 0
		\end{split}
	\end{equation}
\end{subequations}

Or equivalently,
\begin{subequations}
\begin{equation}
b = \frac{\sum_{i=0}^{n}{hum_{n}}}{n} = jeuleu
\end{equation}
\end{subequations}

\section{Conclusion}
Mathieu is nen tank
\newpage

lol
swag
koe
\newline
bobs

$45 + 6 =  \newline 7$

\begin{equation}
lol += equa \newline
zeppers met bobellel zeoln ron osfdn sfdn dfn ndf gnd gofjgoisjgiprz gpir pjrep j = \hbox{} 5 *okpokpojlnonljnoiuoinoinlinoinouinolnouinlknoinljnooooooooooo
\end{equation}
yolo
\hbox{\hbox{}}
\hbox{\newline \newline }
swag
%\bibliographystyle{plain}
%\bibliography{references}
\end{document}
